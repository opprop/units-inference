\documentclass[10pt]{article} 

\usepackage{rotating}
\usepackage{multirow}
\renewcommand{\arraystretch}{1.5}
\usepackage{amsmath}
\usepackage{flexisym}
\usepackage{multicol}
\usepackage{fullpage}
\usepackage{parskip}

% Code
\usepackage{listings}
\usepackage{color}
\definecolor{dkgreen}{rgb}{0,0.6,0}
\definecolor{gray}{rgb}{0.5,0.5,0.5}
\definecolor{mauve}{rgb}{0.58,0,0.82}
\lstset{frame=tb,
  language=Python,
  aboveskip=3mm,
  belowskip=3mm,
  showstringspaces=false,
  columns=flexible,
  basicstyle={\ttfamily},
  numbers=none,
  numberstyle=\tiny\color{gray},
  keywordstyle=\color{blue},
  commentstyle=\color{dkgreen},
  stringstyle=\color{mauve},
  breaklines=true,
  breakatwhitespace=true,
  frame=single,
  tabsize=4
}
 
\let\oldsection\section
\renewcommand\section{\clearpage\oldsection}

\newcommand{\fline}{\noindent\rule{\textwidth}{1pt}}

\makeatletter
\renewcommand*\env@matrix[1][*\c@MaxMatrixCols c]{%
  \hskip -\arraycolsep
  \let\@ifnextchar\new@ifnextchar
  \array{#1}}
\makeatother

\begin{document}

\section{Description}

\subsection{Python Programs}

There are two programs, \texttt{gen_microbenchmarks.py}, which generates a single microbenchmark file, and \texttt{runner.py}, which generates several benchmark files, performs inference on them, and extracts the SMT solving time. 

Use \texttt{python3 gen_microbenchmarks.py --h} to see the complete help menu and what all of the command line arguments mean,
\begin{lstlisting}[identifierstyle=\ttfamily, keywordstyle=\ttfamily]
  -h, --help            show this help message and exit
  --mult MULT           Number of multiplication constraints in each group
  --mult-groups MULT_GROUPS
                        Number of groups of multiplications
  --mult-end            Specify to annotate the end variable of the
                        multiplications in each group
  --mult-perline        Specify to put one multiplication in each line and use
                        intermediate variables to store results
  --mult-nocorrection   Specify to disable using constants to ensure that the
                        number of variables equal the number of
                        multiplications
  --mult-annot MULT_ANNOT
                        Percentage of starting variables for multiplication in
                        each group which should be annotated with some unit
  --add ADD             Number of addition constraints in each group
  --add-groups ADD_GROUPS
                        Number of groups of additions
  --add-end             Specify to annotate the end variable of the additions
                        in each group
  --add-perline         Specify to put one addition in each line and use
                        intermediate variables to store results
  --add-nocorrection    Specify to disable using constants to ensure that the
                        number of variables equal the number of additions
  --add-annot ADD_ANNOT
                        Percentage of starting variables for addition in each
                        group which should be annotated with some unit
  --comp COMP           Number of comparison constraints in each group
  --comp-groups COMP_GROUPS
                        Number of groups of comparison constraints
  --comp-nocorrection   Specify to disable using constants to ensure that the
                        number of variables equal the number of comparisons
  --comp-annot COMP_ANNOT
                        Percentage of starting variables for comparison in
                        each group which should be annotated with some unit
\end{lstlisting}

In particular,
\begin{enumerate}
    \item \texttt{--mult-perline}: There are two ways to multiply $k$ variables together,
    \begin{lstlisting}[language=Java]
    // all in one line
    double result1 = starting1 * starting2 * ... * startingk;
    // one operation per line
    double result1 = starting1 * starting2;
    double result2 = result1 * starting3;
    double result3 = result2 * starting4;
    ...
    \end{lstlisting}
    The default behavior is the first and this flag is used to get the second. 
    \item \texttt{--mult-nocorrection}: To get $k$ multiplication operations, $k + 1$ operands are needed. By default, $k$ starting variables are created (starting variables are those which are initialized to $0$ and later operated on) and the constant $1$ is used as the last operand. This is to correct for the number of variables as the number of groups change, since the variables in each group are independent of the variables in other groups. For example, without correction, 10 groups of 10 multiplications each would result in $(10 + 1)\cdot 10 = 121$ variables whereas 1 group of 100 multiplications would result in 101 variables, even though both settings involve 100 multiplication operations. Use this flag to disable this correction and create $k + 1$ starting variables for a group of $k$ multiplications.
\end{enumerate}
For example, to generate a file with 3 groups of 4 multiplications each, where there is one multiplication per line, no correction, and 75\% of the starting variables are annotated, do: \texttt{\texttt{python3 gen_microbenchmarks.py --mult 4 --mult-groups 4 --mult-perline --mult-nocorrection --mult-annot 75}}. 

\iffalse
\subsection{Number of Slots and Constraints}
Trying to determine the number of slots and constraints as a function of the various settings in the script (number of operations, percentage of annotation, etc). 
\begin{enumerate}
    \item \texttt{constantslot}
    \item \texttt{variableslot}
    \item \texttt{existentialvariableslot}
    \item \texttt{total_slots} $=$ \texttt{total_variable_slots} + \texttt{constantslot}
    \item \texttt{comparableconstraints} is the total number of comparisons
    \item \texttt{arithmeticconstraints} is the total number of additions and multiplications
    \item \texttt{subtypeconstraint}
    \item \texttt{existentialconstraint}
    \item \texttt{preferenceconstraint}
    \item \texttt{equalityconstraint}
\end{enumerate}

Other:
\begin{enumerate}
    \item Initializing a variable (\texttt{double a = 3}) adds two additional \texttt{variableslot} and one additional \texttt{subtypeconstraint}
    \item Initializing a variable and annotating it (\texttt{@m double a = 3}) adds one additional \texttt{variableslot} and one additional \texttt{subtypeconstraint}
\end{enumerate}
\fi

\section{Results}

Averages across three replicates (takes around 1.5-2 hours to run in total) in milliseconds.
\begin{enumerate}
    \item With correction, all operations in one line, not annotating the end result in each group, annotating 100\% of starting variables.
\begin{center}
\begin{tabular}{p{5pt}c|ccccccc}
\multicolumn{2}{l}{} & \multicolumn{7}{c}{Number of Groups} \\
     & & 1  & 5 & 10 & 15 & 20 & 25 & 30   \\ \cline{2-9}
\multirow{7}{*}{\rotatebox[origin=c]{90}{Multiplications per group}}
&  1  & 41 &  37 &  55 &  51 &  60 &  69 & 66 \\
 & 2  & 41 &  41 &  54 &  68 &  75 &  79 & 90\\
  &3  & 42 &  60 &  63 &  77 &  91 &  96 & 112 \\
&   5 & 50 &  134 &  493 &  1402 & 2980 & 5446 & 14193\\
  & 7 & 47 &  207 &  828 &  2689 &  6308 &  17668 & 22678\\
&  10 & 51 &  407 &  2224 &  6459 &  21559 &  38457 & 77021\\
&  15 & 68 &  769 &  5096 &  23430 &  51603 &  183874 & 270886\\ 
\end{tabular}
\end{center}
\begin{center}
\begin{tabular}{p{5pt}c|ccccccc}
\multicolumn{2}{l}{} & \multicolumn{7}{c}{Number of Groups} \\
     & & 1  & 5 & 10 & 15 & 20 & 25 & 30   \\ \cline{2-9}
\multirow{7}{*}{\rotatebox[origin=c]{90}{Additions per group}}
& 1& 37  & 49 &  77 &  113 &  188 &  307 &  525 \\
 & 2& 38  & 62 &  122 &  235 &  426 &  960 &  1304\\
  & 3& 40  & 73 &  164&  363 &  897 &  1296 &  2589 \\
&  5& 45 & 192 &  388 &  1037 &  2069 & 6173 & 9123\\
  & 7& 47 & 188 &  762 &  2045 &  5583 &  11689 &  19466\\
&  10& 54 & 387&  1909&  5426&  13320 &  42211 &  63587 \\
&  15& 56 & 854 &  5048 &  17302 &  74866 &  154856 & 247602\\ 
\end{tabular}
\end{center}
\item With correction, all operations in one line, annotating the end result in each group (arbitrarily for multiplication, which possibly causes inference failure, and correctly for addition), annotating 75\% of starting variables.
\begin{center}
\begin{tabular}{p{5pt}c|ccccccc}
\multicolumn{2}{l}{} & \multicolumn{7}{c}{Number of Groups} \\
     & & 1  & 5 & 10 & 15 & 20 & 25 & 30   \\ \cline{2-9}
\multirow{7}{*}{\rotatebox[origin=c]{90}{Multiplications per group}}
&  1  & 31 &  50 &  67 &  80 &  125 &  163 & 241 \\
 & 2  & 42 &  53 &  128 &  243 &  389 &  579 & 1134\\
  &3  & 39 &  91 &  215 &  508 &  1027 &  1694 & 2651 \\
&   5 & 62 &  197 &  686 &  1759 & 3628 & 5605 & 9801\\
  & 7 & 56 &  296 &  1062 &  3283 &  6845 &  12501 & 20661\\
&  10 & 71 &  447 &  2359 &  7095 &  14630 &  32354 & 58695\\
&  15 & 76 &  942 &  5522 &  18594 &  49394 &  86678 & 174256\\ 
\end{tabular}
\end{center}
\begin{center}
\begin{tabular}{p{5pt}c|ccccccc}
\multicolumn{2}{l}{} & \multicolumn{7}{c}{Number of Groups} \\
     & & 1  & 5 & 10 & 15 & 20 & 25 & 30   \\ \cline{2-9}
\multirow{7}{*}{\rotatebox[origin=c]{90}{Additions per group}}
& 1& 40  & 42 &  59 &  54 &  63 &  62 &  67 \\
 & 2& 45  & 44 &  56 &  70 &  78 &  82 &  95\\
  & 3& 45  & 47 &  67&  72 &  84 &  103 &  114 \\
&  5& 37 & 56 &  89 &  102 &  140 & 138 & 165\\
  & 7& 42 & 60 &  108 &  138 &  157 &  205 &  234\\
&  10& 52 & 84&  127&  225&  236 &  247 &  297 \\
&  15& 44 & 112 &  234 &  320 &  396 &  424 & 439\\ 
\end{tabular}
\end{center}
\item With correction, all operations in one line, annotating the end result in each group, annotating 100\% of starting variables.
\begin{center}
\begin{tabular}{p{5pt}c|ccccccc}
\multicolumn{2}{l}{} & \multicolumn{7}{c}{Number of Groups} \\
     & & 1  & 5 & 10 & 15 & 20 & 25 & 30   \\ \cline{2-9}
\multirow{7}{*}{\rotatebox[origin=c]{90}{Multiplications per group}}
&  1  & 32 &  33 &  37 &  35 &  41 &  35 & 39 \\
 & 2  & 39 &  31 &  35 &  28 &  33 &  36 & 44\\
  &3  & 32 &  38 &  37 &  44 &  40 &  47 & 53 \\
&   5 & 36 &  36 &  47 &  50 & 51 & 61 & 63\\
  & 7 & 36 &  35 &  47 &  55 &  72 &  75 & 78\\
&  10 & 34 &  52 &  63 &  80 &  96 &  96 & 112\\
&  15 & 37 &  63 &  76 &  103 &  127 &  148 & 187\\ 
\end{tabular}
\end{center}
\begin{center}
\begin{tabular}{p{5pt}c|ccccccc}
\multicolumn{2}{l}{} & \multicolumn{7}{c}{Number of Groups} \\
     & & 1  & 5 & 10 & 15 & 20 & 25 & 30   \\ \cline{2-9}
\multirow{7}{*}{\rotatebox[origin=c]{90}{Additions per group}}
& 1& 35  & 36 &  50 &  56 &  64 &  69 &  72 \\
 & 2& 36  & 44 &  63 &  62 &  64 &  83 &  96\\
  & 3& 33  & 56 &  74&  84 &  88 &  102 &  111 \\
&  5& 32 & 55 &  84 &  100 &  125 & 139 & 164\\
  & 7& 37 & 67 &  91 &  137 &  150 &  182 &  216\\
&  10& 39 & 86&  135&  167&  205 &  263 &  319 \\
&  15& 48 & 97 &  216 &  296 &  330 &  400 & 455\\ 
\end{tabular}
\end{center}
\end{enumerate}


\end{document}

